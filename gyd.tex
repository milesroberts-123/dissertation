
\begin{itemize}

\item Some future directions: train a generalized selective sweep model and fine tune it for particular circumstances?
    
\item One way forward is through time series data \citep{whitehouse_timesweeper_2023}, but this can be limited by the time and funding of researchers.

\item \citep{ormond_inferring_2016}

\item \citep{flagel_unreasonable_2019}

\item \citep{ormond_inferring_2016}

\item \citep{torada_imagene_2019}

\end{itemize}

Given a candidate beneficial mutation, as might be identified in mapping or selection scan studies, two key questions to ask are: (1) what was the strength of selection that acted on the mutation and (2) what was the timing of when selection began acting on the mutation \citep{he_estimation_2020, bisschop_sweeps_2021}? Another way to rephrase these questions is to ask (1) what was the fixation time ($t_f$) of the beneficial mutation and (2) how long ago prior to sampling did the beneficial mutation fix ($t_a$) (Figure \ref{fig:ident})? The sum of these two timescales ($t_f + t_a$) describes the timing of selection on the beneficial mutation and $t_f$ reflects the strength of selection that acted on the mutation (because $tf \propto 1/s$). However, disentangling $t_f$ and $t_a$ from a single time-point observation is difficult. Some of the signatures of a rapid $t_f$ are degraded after about only $0.4N_e$ generations \citep{przeworski_signature_2002}. There is also an identifiably problem where multiple combinations of $t_f$ and $t_a$ can result in similar outcomes (Figure \ref{fig:ident}B). As the total length of an allele frequency trajectory ($t_f + t_a$) increases, there is more time for mutation and recombination to break up the sweep. This means that old, fast (high $t_a$, low $t_f$) sweeps will produce similar outcomes as young, slow (high $t_a$, low $t_f$) sweeps (Figure \ref{fig:ident}B). The ease of predicting $t_a + t_f$ compared to predicting $t_a$ and $t_f$ individually could be one reason why coalescent times receive greater focus.


\begin{itemize}
    \item \citep{fisher_xxidominance_1923}
    \item \citep{haldane_mathematical_1927}
    \item \citep{wright_evolution_1931}
    \item \citep{darling_first_1953}
    \item \citep{feller_diffusion_1954}
    \item \citep{watterson_theoretical_1962}
    \item \citep{ewens_mean_1963}
    \item time to fixation for a neutral mutation is $4N_e$ generations on average  \citep{kimura_average_1969}
    \item \citep{kimura_average_1980}.
    \item \citep{kimura_probability_1970}
    \item \citep{kimura_probability_1974}
    \item \citep{burrows_distributions_1974}
    \item \citep{ohta_time_1983}
    \item \citep{przeworski_signature_2002}
    \item \citep{przeworski_estimating_2003}
    \item \citep{malaspinas_estimating_2012}
    \item \citep{chen_inferring_2013}
    \item good review of math \citep{otto_fixation_2013}
    \item \citep{greven_fixation_2016}
    \item \citep{ormond_inferring_2016}
    \item \citep{schraiber_bayesian_2016}
    \item \citep{stern_approximate_2019}
    \item \citep{nakagome_inferring_2019}
    \item time to fixation in changing environments \citep{cui_fixation_2018, kaushik_time_2021}
\end{itemize}

A key determinant of a mutation's fate is its effect on fitness. If we can measure the fitness effects of a mutation, then one of three outcomes are generally expected. First, if the mutation is ``bad", meaning it decreases an organism's ability to survive and reproduce, then we would expect the mutation to eventually be selected out of the population. If the mutation is neutral, then we would expect the mutation to be inherited randomly before being either lost or fixed. Finally, ``good" mutations, will be passed on at a higher rate than other mutations and so will eventually fix (assuming they escape Haldane's sieve).





Many studies of beneficial mutations seek to understand the selection coefficient and coalescence times of the mutations (XXX). The main benefit of understanding times to fixation though is that they can provide more historical context for when selection began acting on a beneficial mutation. Given a selective sweep region, can we predict how long it took to fix? It is uncommon for studies to estimate times to fixation for fixed beneficial alleles.



Assuming that a mutant is not lost by drift, the time to fixation for a neutral mutation is $4N_e$ generations on average \citep{kimura_average_1969}, and is proportional to $1/s$ for beneficial mutations \citep{otto_fixation_2013}. 

A few core results have emerged. If one copy of a neutral mutant allele is introduced to a population each generation, then the mutant allele will take $4N_e$ generations to fix on average (assuming that the mutant is not lost by drift) \citep{kimura_average_1969}. For positively selected mutations in a constant finite population, the time to fixation is proportional to $1/s$ \citep{otto_fixation_2013}. Conditional on fixation, deleterious and beneficial mutations have the same mean time to fixation \citep{otto_fixation_2013}. The probability that a beneficial allele reaches fixation is fairly small \citep{haldane_mathematical_1927, zhao_characteristic_2013}. Since this foundational work, focus has shifted to understanding time to fixation under more complex demographic scenarios, including population subdivision and changing environments. 




Reasons to care about $t_f$

\begin{itemize}
    \item tf is the inverse of the adaptation rate?
    \item tf differentiates hard vs soft sweeps? (softer = longer tf)
    \item tf alters the signals of sweeps, the longer a sweep takes, the more time recombination has to break up haplotype structure, reducing classic selective sweep signatures
    \item tf implies limits to selection, in the case of non-polygenic adaptation, how quickly the allele fixes
\end{itemize}




Traditionally, scientists would first hypothesize a model of a selective sweep, then hypothesize what data that model would generate, and finally devise test statistics to measure sweep signatures. A more recent example is \citet{monnahan_effect_2020, hartfield_selective_2020}. The authors ask what sweep signatures might be unique to high ploidy and self-fertilizing species, respectively.   

Machine learning now allows us to reverse this process. Instead of starting with theory and hypothesizing what signatures sweeps create, we can instead generate lots of data, train a model to search for patterns in the data, then attempt to interpret the signatures the models use to make their predictions.

Understanding factors influencing time to fixation has been given extensive theoretical treatment. However, there are no methods to predict fixation times.

The main factor that affects the time to fixation is the selection coefficient. However, other factors play an important role: population structure, population growth, mating system.


Whether we want to understand how organisms respond to climate change, how to grow more food, or human health, one player is at the center of it all: mutation. 

The process of mutation plays a role in from human health and cancer, to climate change, to agriculture. 